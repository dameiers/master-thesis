\chapter{Technology Analysis}\label{chap:tech_analysis}
	
\section{Choosing Java Browser Components}\label{chap:browser_api_comparison}
There are already existing Java APIs that allows the visualisation of HTML applications in Java Swing components.
They all offer the basic functionialty to parse HTML documents and visualise them in Swing components.

The following APIs are examined with regard to their ability to visualize modern web applications, which means applications that highly make use of JavaScript, CSS3 and HTML 5 features.
In particular the above introduced requirements for browser APIs are taken into consideration.

In the current implementation of the cids navigator it is already possible to use simple HTML sites as a description page for catalogue nodes.
It is possible to choose  between two different HTML rendering API's, Flying Saucer and Calpa HTML.
They are examined first in regard for a possible reuse.
 

% Calpa HTML
\textbf{CalpaHTML} is the first component that can be used to visualise html documents as a description page for nodes in the cids catalogue.
Despairingly, it seems that the project was abandoned because no further information (API Documentation, feature list etc) can be found. 
Past experiences with CalpaHTML have shown the very limited feature set.
Thus, Flying Saucer as an optional HTML rendering component can be used.

% Flying Saucer
\textbf{Flying Saucer} is an XHTML renderer that allows the rendering of XML and XHTML documents to images, pdf documents and Swing components such as JPanels or ScrollPanes as well.
The main idea of Flying Saucer is to parse any kind of XML documents and to apply CSS styles to them.
According to the website, Flying Saucer supports nearly all CSS 2.1 with a few exceptions and additionally "includes a small handful of CSS 3 features" \autocite{tech-ana:flying-saucer}.
Unfortunately it is not possible to determine which CSS3 features are supported and which not in more detail.
Documents parsed with Flying Saucer need to be valid and standard compliant documents.
It is not possible to parse faulty documents.
The developers recommend the usage of an HTML cleaner like TagSoup or JTidy, which are able to make invalid HTML documents valid.
The documentation of Flying Saucer is good.
There are several How-To articles as well as an developer guide that makes it easy to use Flying Saucer.
Furthermore there is a mailing list where it is possible to get support from the developers and the community.
Unfortunately, there is a rapid decrease in the activity of the mailing group since 2010.
The most disqualifying factor is that Flying Saucer lacks in supporting JavaScript or any kind of user editable content, since the parsed documents are read only.

% the Lobo Project

The \textbf{Lobo Project} offers a complete stand alone Java based web browser as an alternative to other mdoern browsers.
It consists of the Cobra toolkit which is responsible for HTML parsing and rendering on the one side and a browser application on the other side.
The devlopers emphasize that the Lobo browser is an "[...]  browser application that has been documented so it can be used as an API." \autocite{tech-ana:lobo} and that sits on top of the Cobra toolkit.
It enriches its users with additional functionality such as navigation, request handling, authentication, caching and so on.
The Cobra HTML renderer and parser fully supports HTML 4, Javascript and CSS 2.
HTML5 features and CSS3 are not supported.
Both projects are available under a LGPLv2 license.
The API is well documented, but there are only a few tutorials and how-to documents which makes it a bit harder to start.
This is especially true for the Cobra toolkit.
There is a help forum available but sparsely used and maintained, hence there are a lot of unanswered questions.
This and the fact that the last release was in 2009 indicates that the project is shelved.


% WebRenderer

Another full Swing and embedded browser component is \textbf{WebRenderer}.
It is built upon the Mozilla technology and therfore supports an impressive list of standards and features such as many HTML5 features, CSS1 and 2, JavaScript, Java Applets, SVG and MathML just to name a few.
Another outstanding point in contrast to the earlier discussed frameworks is the fine granular control over the loaded web page.
WebRenderer gives full access to the DOM of the loaded site and offers a full event model which allows it to react on mouse events, browser events, network events and DOM events.
The API is well documented and there are a lot of code Examples as well as an developer guide.
The drawbacks of WebRenderer are a missing CSS3 support and, the more serious issue, the license.
WebRenderer is commercial product and can therefore not be used.


% Java FX WebView

%ToDO: explain what is JavaFX...
In \textbf{JavaFX} there is a built in browser component called WebView, that is based on WebKit an open source browser engine, which is also used in modern browsers like Safari.
Similar to WebRenderer there is also a high support of modern web standards.
To give an example, WebView supports the following HTML5 features: Canvas, MediaPlayback, form controls, editable content and so on.
Furthermore it supports JavaScript and several CSS3 features.
It also offers a bi-directional communication between JavaScript code of the loaded site and JavaFX.
Hence WebView is a native JavaFX component, the documentation as well as the support are very good and it is no problem to find information or get help on a special problem.
The JavaFX community is large and growing, and there are several forums, among others the Oracle JavaFX forum.
Since Java 7, JavaFX is part of the standard Java runtime environment and hence available platform indepent.
In contrast to the Lobo project or CalpaHtml, it is unlikely that the support of the WebView component gets discontinued.

All this make the JavaFX WebView to the best suiting candidate, it supports an extended set of HTML5 features, JavaScript and CSS3, is available as open source and is well documented with a large and active community.
But the usage of JavaFX emerges the problem to integrate JavaFX components in the cids Navigator respectively swing components.
Fortunately JavaFX also provides a component for this task.
The JFXPanel is a Swing component that acts as a JavaFX runtime container and makes it possible to run arbitrary JavaFX application scenes in it.
 

The following evaluation matrix summarises the pros and cons of the above introduced browser components and APIs. Disqualifying factors are marked with a coloured background

\input{chapters/4_tech_analysis/table_browser_apis}

\subsection{Testing and Comparing the JavaFX WebView}\label{chap:browser_api_testing}

Despite an exhaustive literature research it was not possible to find a full feature list of the WebView concerning modern web standards like CSS3, JavaScript or HTML5.
To ensure that the WebView is able to render modern web applications, especially JavaScript based applications, correctly, it is necessary to test the compliance to web standards and the behaviour when trying to visualise JavaScript applications.
This is done with several test tools that are available on the web.
The following section carries out what test tools are used, gives a short description about each test tool and outlines the results of the WebView as wells as the results of the Google Chrome browser to allow a first comparison.

The first test that is executed is the ACID 3 test originated in 2008.
The test is defined by the Web Standards Project and can be found at \autocite{tech-ana:acid3}.
According to \autocite{tech-ana:acid3} the test \enquote{[...] is primarily testing specifications for ''Web 2.0'' dynamic Web applications}. A full list of tested standards can be found at \autocite{tech-ana:acid3} but among others the CSS3 Color, CSS3 UI, SVG and a lot of DOM standards are tested.
Important to note is, that the resulting image is not the only criteria to pass the test.
Also the animation has to be smooth.
As figure \ref{fig:acid_webview} depicts, the test is not fully passed.
Unfortunately the test gives no information what test has failed or what standard the browser is not compliant to. \todo{give . Finde what hier komisch, which ist passender.}
Besides this obvious failure, the animation smoothness compared to other browsers, especially compared to Google Chrome, is considerably worse.
Another important thing to mention is, that it is recommended to execute the ACID 3 test with default settings hence e.g changing the zoomlevel "[...] may alter the rendition of the test page without this constituting a failure in compliance." \autocite{tech-ana:acid3}. \todo{enquote benutzen}
In fact the tests are executed with the Oracle demo application, where it is not possible to a maximum size \todo{maximize?} for the application, it is unclear if the layouting mechanism used in JavaFX influences the test results.

\begin{figure}
\begin{minipage}[t]{7cm}
	\centering
	\includegraphics[width=7cm]{./img/tech-ana/acid_result_webView.png}
	\caption{Result Image of the Acid Test - WebView browser}
	\label{fig:acid_webview}
\end{minipage}
\hfill
\begin{minipage}[t]{7cm}
	\centering
	\includegraphics[width=7cm]{./img/tech-ana/acid_reference.png}
	\caption{Reference Image of the Acid Test}
	\label{fig:acid_reference}
\end{minipage}
\end{figure}

%HTML5 test 

To test the abilities of the WebView regarding HTML5 support, the HTML5 test \autocite{tech-ana:html5-test} is executed.
The test is seperated in different features that are part of the upcoming HTML5 standards as well as drafts and specifications.
It tests how many of these features are already supported and calculates a score.
Important to note is that the feature set that is tested is not complete and that the test does not check the correctness of the implementation.
As already mentioned, a lot of the HTML5 standards are still under development and could change before receiving a final status.
Thus, the rather poor result (see \ref{fig:html5_webview}) can be relativated a bit.

\begin{figure}
\begin{minipage}[t]{7cm}
\todo{Falsches Caption}
	\centering
	\includegraphics[width=7cm]{./img/tech-ana/html5_result_webview.png}
	\caption{Result Image of the Acid Test - WebView browser}
	\label{fig:html5_webview}
\end{minipage}
\hfill
\begin{minipage}[t]{7cm}
	\centering
	\includegraphics[width=7cm]{./img/tech-ana/html5_result_chrome.png}
	\caption{Reference Image of the Acid Test}
	\label{fig:html5_chrome}
\end{minipage}
\end{figure}

% CSS3 test

As already mentioned in the requirements, it is vital that the browser supports as much CSS 3 features as possible.
Hence it is not possible to find information regarding the CSS3 compatibility of the WebView, the CSS3 \autocite{tech-ana:css3-test}is used to get a better insight on this issue.
Similar to the HTML5 test, the test is divided into the different modules defined by the CSS3 standard.
Again, the test checks only the support of the various features rather the correct implementation.
Figure \ref{fig:css3_webview} depicts the result of test.
In this case, the result of the WebView is comparable with the result of Google Chrome (see figure \ref{fig:css3_chrome}).
The reason for the general poor results might be that the CSS standard is still under development.

\begin{figure}
\begin{minipage}[t]{7cm}
	\centering
	\includegraphics[width=4cm]{./img/tech-ana/css3_webview.png}
	\caption{Result Image of the Acid Test - WebView browser}
	\label{fig:css3_webview}
\end{minipage}
\hfill
\begin{minipage}[t]{7cm}
	\centering
	\includegraphics[width=4cm]{./img/tech-ana/css3_chrome.png}
	\caption{Reference Image of the Acid Test}
	\label{fig:css3_chrome}
\end{minipage}
\end{figure}


%JavaSCript -> ECMA 262 Test Suite:
%Wikipedia: Test262 is an ECMAScript conformance test suite that can be used to check how closely a JavaScript implementation follows the ECMAScript 5th Edition Specification.
The test suite contains thousands of individual tests, each of which tests some specific requirements of the ECMAScript specification.

The last thing to test are the JavaScript abilities.
The ECMA International organisation, which is responsible for the standardisation of the ECMAScript language (the official name of JavaScript) provides the \emph{Test262} test suite that can be used \enquote{[..] to check how closely a JavaScript implementation follows the ECMAScript 5th Edition Specification} \autocite{tech-ana:ECMAScript}.
It provides thousands of tests \enquote{[...]each of which tests some specific requirements of the ECMAScript Language Specification.} \autocite{tech-ana:ECMAtest}.
Although the suite consist of a huge amount of test cases, the authors emphasise that \enquote{test262 is not yet complete.
It is still undergoing active development.} \autocite{tech-ana:ECMAtest}.
In contrast to the above mentioned tests, the ECMA test262 suite lists every single failed test and offers additional information to this test case.
Thus, it provides a good documentation of the bugs that are shipped with the tested JavaScript environment.
Regarding the test result of the WebView, there are 236 test cases that fails.
Although this seems to be a large number, the percentage stake of failed tests is about 2\%, which is a good result.
   
\begin{figure}
\begin{minipage}[t]{7cm}
\todo{falsches caption}
	\centering
	\includegraphics[width=7cm]{./img/tech-ana/ecma_result_webview_small.png}
	\caption{Result Image of the Acid Test - WebView browser}
	\label{fig:ecma_webview}
\end{minipage}
\hfill
\begin{minipage}[t]{6cm}
	\centering
	\includegraphics[width=6cm]{./img/tech-ana/ecma_result_chrome_small.png}
	\caption{Reference Image of the Acid Test}
	\label{fig:ecma_chrome}
\end{minipage}
\end{figure}

\section{Comparison of modern Java Script Frameworks}

\todo{hier sollte noch ein satz hin. überschrift nach überschrift sieht doof aus.}

\subsection{Overview}

There is a huge and steadily increasing amount of new JS MVC frameworks.
A good overview over existing projects is provided by the Todo MVC App \enquote{tech-ana:todo-mvc}.
The Todo MVC app was created with the intention to help developers selecting a JavaScript framework, by providing implementations of the very simple and ever same ToDo App, which allows to define a set of tasks that can be marked as done or undone.

%The application allows to add/delete/edit todo tasks and to mark/unmark tasks as done/undone.

The TodoMVC app gives also a good evidence of the rapid increase of existing JavaScript MVC frameworks.
Figure \ref{fig:todoMvcComp} shows a screenshot of the ToDoMVC taken at two different dates with a period of a year.

\begin{figure}
	\centering	\includegraphics[width=1.0\textwidth]{./img/tech-ana/todo_mvc_app_comparison.png}
	\caption{Screenshot TodoMVC app left:July 2012, right: December 2013}
	\label{fig:todoMvcComp}
\end{figure}

The large amount of different frameworks makes it impossible to examine them all in detail.
Therefore a two step approach to filter the best suiting candidate is applied.
In a first step a very rough comparison of the frameworks is made to filter out the most inappropriate frameworks.
The remaining frameworks are then examined in more detail in chapter \ref{chap:detail_comparison}.
Furthermore a simple application is implemented with each framework that is selected for the detailed comparison to get a better understanding of the concepts used there.
The implemented application is a Simple Greeting Application where a user can edit its first and last name and sees a greeting. 
It is obvious that the complexity of this demo application is not sufficient to demonstrate the powerfulness of the frameworks or to compare them. But it allows first impressions of the concepts that are used, and how effectively writing boilerplate code is avoided.

%The implemented application is a simple cids renderer for an example vermessungriss bean taken from wunda.
%This example was choosen since such renderer are one part that will be implemented in future.
   
The following list shows relevant JS MVC frameworks taken from the ToDoMVC app.
The list only contains frameworks which are available in a stable release (at least 1.0.0) and available under a GPL compatible license.
The TodoMVC App also lists combinations of different frameworks which are also excluded in the following list.\\

\begin{itemize}
\begin{minipage}{0.5\textwidth}

	\item Dojo
	\item YUI
	\item CanJS
	\item soma.js
	\item Maria
\end{minipage}
\begin{minipage}{0.5\textwidth}

	\item Backbone JS
	\item AngularJS
	\item Ember JS
	\item Knockout JS
	\end{minipage}
\end{itemize}


Dojo as well as YUI (Yahoo! User Interface) are not MVC frameworks in the proper meaning of an MVC Framework.
Similar to JQuery , they are rather DOM/Ajax wrapping libraries which is a more suitable description for those frameworks.
Both has several additional features like UI-elements, effects and animations.
Both doesn't provide components that implement MVC pattern and don't provide the user with built in features like data binding, templating or routing and are therefore not taken into further considerations.

CanJS, Soma.js and Maria are very new MVC Frameworks all providing a different set of features and a different emphasis.
According to \autocite{tech-ana:canJS-presentation} CanJS combines the best features of lightweight frameworks like BackboneJS, which are easy to learn and have a small size, as well as of heavy frameworks like EmberJS which offer a lot of good features like live binding, computed properties and memory safety.
CanJS can use EJS or Mustache as templating engines which are both string based templating engines.
In one sentence CanJS can be characterized through its high performance, its memory leak prevention and the fact that it can be used with a lot of different DOM libraries.

Although soma.js can be used as a MVC framework, the basic idea behind soma.js is a more general and architectural approach.
"soma.js  provides tools to create a loosely-coupled architecture broken down into smaller pieces." \autocite{tech-ana:somaJS} \todo{enquote} In order to do so, it implements a large set of different design patterns, like dependency injection, observer pattern or mediator pattern and the MVC pattern.
Soma.js implements its own template engine based on soma.js.
The more architectural bias makes it less comfortable for the developer to work with soma.js since a lot of work needs to be done manually that other frameworks do automatically.

The authors of Maria emphasize that the framework implements the "real" MVC pattern.
Maria is a very lightweight framework which means that there is no built in support for templates and data binding.
Similarly to soma.js, working with Maria means to write a lot of boilerplate code.

Generally speaking, the biggest problem with the above mentioned very young frameworks is, that these projects are not so well documented and there is only a small community which makes it very difficult to find information or help when problems occur.
This disqualifies these frameworks as possible candidates.

One of the more matured and proven frameworks is Backbone JS.
A lot of really large and impressive projects are built with backbone.js such as LinkedIn, AirBnB, Trello or FourSquare.
Hence its maturity, it is well documented and there is a large and active community which makes it easy to find help.
Backbone JS \todo{Backbone JS sollte nicht umgebrochen werden} consists of different components concerning all necessary aspects of building single page web application such as models, views and routers.
Thus application built with backbone normally have a clear defined structure \autocite{tech-ana:heise-backbone}.
Backbone has built in support for persisting model data to REST apis. \todo{API gross? siehe auch unten}
To make this built in component to work properly it is necessary to have a correlation of your model names and the REST api paths.
The backend connection can be implemented manually if the standard one is not suited.
There are already some existing Modules that do this like Backbone.localstorage which stores all data in the Browser itself. \todo{nomen wurden hier grossgeschrieben}

An adverse is, that Backbone JS is not able to reflect changed model data to the view automatically (and vice versa).
Backbone's approach to achieve this is much more lightweight.
Each model can fire a set of events.
A backbone view can listen to these events.
Each view has a render function, which generates the HTML for displaying the view.
This function needs to be implemented manually for each view.
This approach allows it to use any templating engine.
Since backbone needs the dependency to underscore \todo{was ist underscore?}, you can use the templating function from this library.
Another drawback is that, backbone does not manage the display of different views.
Admittedly it provides a router object, but again, this objects just defines what function should be executed when a route is loaded.
The developer has to initialize the new view and to destroy all views no longer needed.
There is always the danger of memory leaks which causes several performance and security issues.
At least, using backbone \todo{Backbone JS} means a lot of initialisation overhead, since it is first necessary to define the domain model and set up the infrastructure.
Considering all this disqualifies Backbone JS as possible candidate.
      
The remaining 3 Frameworks does not have any of the above mentioned drawbacks and are used for a much more detailed examiniation described in the following chapter.

\subsection{Comparison of AngularJS, EmberJS and KnockoutJS}\label{chap:detail_comparison}

\todo{hier auch ein kleiner satz?}

\subsubsection{Knockout JS}

Knockout JS is a more lightweight library, which emphasis is to implement the MVVM (Model-View-ViewModel) pattern for HTML.
This pattern enriches HTML with two way data binding and a seperation of concerns, which means that the application logic is seperated from the business logic and the model.

The two way data binding is implemented by using special Java Script Objects, so called Knockout observables and a special HTML attribute both provided by Knockout.
Thus model changes are automatically reflected to the view.
Listing \ref{lst:ko-1} demonstrates how to bind a property of a ViewModel to an HTML textfield element.

\begin{lstlisting}[label=lst:ko-1,caption=the ViewModel]
function ViewModel(){
	this.firstName = ko.observable('Daniel');
	this.lastName = ko.observable('Meiers');
	this.fullName = ko.computed(function({
				return this.firstName()+" "+this.lastName();
			},this); 
}
ko.applyBindings(new ViewModel());
\end{lstlisting}

\begin{lstlisting}[label=lst:ko-2,caption=the html view]
 	<div>
		<label>First Name</label> 
		<input type="text" data-bind="value: firstName">
	</div>
	<div>
		<label>Last Name</label> 
		<input type="text" data-bind="value: lastName">
	</div>
	<p data-bind="text: 'Hello '+fullname"></p>
\end{lstlisting}

The data-bind attribute of the input element (see line 3 in listing \ref{lst:ko-2}) tells Knockout to bind the properties of the ViewModel object to an attribute of the containing element (in this case the value attribute of the input element).
There is a large set of predefined bindings for different purposes.
There \todo{They? - Ich habe Probleme den Satz zu verstehen.} are controlling and appearance bindings such as the visible or the style binding, control flow bindings such as the foreach binding to iterate over a set of ViewModel properties and at least, form field bindings.
If these bindings are not sufficient it is possible to create custom bindings.


The last line in listing \ref{lst:ko-1} binds the ViewModel to the view.
After that all changes in the ViewModel are reflected to the input element and vice versa.
The ViewModel itself defines two different sort of properties.
The properties 'firstName' and 'lastName' are normal observables.
The 'fullName' property is a computed observable which can be any calculated value, also considering other observables.
Since the properties are special Knockout objects, it is necessary to access the property values like a function (see line 5 in listing \ref{lst:ko-1}).

\begin{figure}[h]
	\centering	
	\includegraphics[width=0.7\textwidth]{./img/tech-ana/mvvm.jpeg}
	\caption{MVVM pattern, source \autocite{heise_knockout}}
	\label{fig:mvvm}
\end{figure}

Knockout also provides a more convenient way to define the properties of a ViewModel, the mapping plugin.
The mapping plugin converts any property of any JSON object to a Knockout observable in order to act as ViewModel.
The following listing (\ref{ko_mapping}) shows the above introduced name example with the mapping plugin.

\begin{lstlisting}[label=ko_mapping,caption=the mapping plugin]
var person = {
	firstName : 'Daniel';
	lastName : 'Meiers';
	fullname : function({
			return firstname +lastName;
		});
var viewModel = ko.mapping.fromJS(person);
\end{lstlisting}

Knockout JS doesn't support attaching a server side backend or REST API.
Furthermore it is not able to route between different pages of your application.
Thus these parts of the SPA \todo{was ist SPA?} has to be implemented by your own or with additional frameworks.

\subsubsection{Ember JS}

Ember JS is one of the younger frameworks, mainly created by Yehuda Katz and Tom Dale and introduced in December 2011 \autocite{tech-ana:announcing-ember}.
In this short time Ember JS could make a really impressive progress and has gained a lot of attraction which can be concluded from the number of github stars and watches. \todo{GitHub gross schreiben. Sternchen zählen finde ich nicht so wissenschaftlich. Evt. umschreiben.}
It comprises all necessary parts for building web applications including two way data binding with templates, a routing mechanism and provides a way to connect the application to REST API's.

Ember \todo{Einheitliche Schreibeweise einhalten? Ember.js kommt auch vor. Bei Ember JS auf zeilenumbruch achten. Nur Ember Benutzen ist wohl auch ok, besonders später. Dann aber im groß.} is a very stringent and opinionated framework and is built upon concepts such as DRY (dont \todo{don't?} repeat yourself) and CoC (Convention over Configuration).
Especially the usage of naming conventions allows it to write applications with a nominal amount of code.
The disadvantage of this is, that it is more sophisticated to get started with Ember.js because the philosophy and the concepts of Ember need to be understand in first step.
The full stack and opinionated approach are the reason why Ember is one of the largest frameworks (56k).

It uses Handlebars as its templating engine, which is a very popular string based templating engine.
Ember.js also provides a way to connect your application to RESTful WebServices with an additional project ember-data.
As well as Ember itself, ember-data highly makes use of the Convention over Configuration principle which means that ember-data expects the data provided by a REST-Service in a special format.
The Ember team substantiate : "[...]we don't think most web developers should have to write any custom XHR code for loading data.
Strong conventions on the client and strong conventions on the server should allow them to communicate automatically." \autocite{tech-ana:ember-data}.
Besides Ember-Data it is also possible to use any other server connection implementation.

To build an Ember application it is first necessary to create an instance of an Ember.Application object (see listing \ref{em_js}).
\begin{lstlisting}[label=em_js, caption=app.js]
var App = Ember.Application.create();
\end{lstlisting}

\begin{lstlisting}[label=em_html,caption=index.html]
<html>
    <head>
        <title></title>
        <meta http-equiv="Content-Type" content="text/html; charset=UTF-8">
    </head>
    <body>

    </body>
    
    <script type="text/x-handlebars">
	<div>
	{{outlet}}
	</div>
    </script>

    <script type="text/x-handlebars" data-template-name="index">
	<div>
		<label>First Name</label> 
		<input {{bindAttr value=firstName}}></input>
	</div>
	<div>
		<label>Last Name</label> 
		<input {{bindAttr value=LastName}}></input>
	</div>
	<p>Hello {{fullName}}</p>
   </script>

    <script src="js/libs/jquery-1.9.1.js"></script>
    <script src="js/libs/handlebars-1.0.0-rc.4.js"></script>
    <script src="js/libs/ember-1.0.0-rc.6.js"></script>
    <script src="js/app.js"></script>
</html>
\end{lstlisting}

Creating the application object defines the namespace of the application.
All other classes, like Routers, Controllers, Views or Models are defined as properties of the application object.
This has the advantage that the JavaScript namespace does not get polluted by application depended objects.
Besides some other initialisation, the create() statement automatically creates a default router for the application.
The default router first sets up the ApplicationRoute with the ApplicationTemplate, a pre-defined and special route only for application startup.The ApplicationTemplate is defined in listing \ref{em_html} in line 10.
By convention Ember treats this as Application template in fact the id attribute of the script tag is missing.

The ApplicationTemplate is the right place for static content like headers, footers or menu bars, in fact it gets always rendered first.
The \{\{outlet\}\} tag tells ember where to fill in dynamic content that depends on the currently loaded route.
After this initialization the Ember router starts the routing process and routes to the current route.
This is normally the plain basic url. \todo{URL gross, auch unten}
This path is connected per convetion to Embers IndexRoute so that the Router takes all steps to initialize the IndexRoute.
One important note to mention here, is that the templates are standard handlebars templates. \todo{Handlebar Name und Einzahl?}

As outlined above, a very central concept of Ember is routing.
Every Ember application needs a Router in fact the router translates a URL into a series of templates and is responsible for loading these templates as well as respective model data and sets up other application state.
Listing \ref{lst:em_router} demonstrates a Router definition with two different routes.

\begin{lstlisting}[label=lst:em_router,caption=Router Defintion in Ember]
App.Router.map(function() {
  this.route("about");
  this.route("favorites");
});
\end{lstlisting}

Embers routing mechanism is good example how the CoC apporach is used in Ember to avoid boilerplate code. As listing \ref{lst:em_router} depicts, only the routes itself are defined but no programming logic that defines how to set up the routes. 
This is because a naming convention is used to reason from the current url to the route name and from the route name to the controller and the template to display.
The table in figure \ref{fig:ember_routes} demonstrates the naming convention.

\begin{figure}
	\centering	\includegraphics[width=1.0\textwidth]{./img/tech-ana/ember_routes_conventions.png}
	\caption{Ember Conventions for Routes, source \autocite{tech-ana:em-routes-coc}}
	\label{fig:ember_routes}
\end{figure}

Starting the above mentioned code snippets would result in a page showing empty input fields for both properties, because the templates that are involved are not backed with a model that contains data the templates should display.

Every template is backed by a model, and the route defines what model should be used by the template.
As depicted in the table above, ember will look for an App.IndexRoute by convention, so this is the right place where we need to point to the model.

\begin{lstlisting}[label=em_controller,caption=app.js]
var App = Ember.Application.create();
App.Person = Ember.Object.extend({
	firstName: null,
	lastName : null,
    fullName: function() {
        return this.get('firstName') + " " + this.get('lastName');
    }.property('firstName','lastName')
});

var myself = App.Person.create({
    firstName: "Daniel",
    lastName: "Meiers",
});

App.IndexController = Ember.ObjectController.extend({});

App.IndexRoute = Ember.Route.extend({
    model: function() {
        return myself;
    }
});
\end{lstlisting}

Listing \ref{em_controller} needs some more explanation.
First at all we define a model class called Person which has the same properties as in the Knockout example.
In order to do this, we need to extend a special ember class in fact we want to bind to these properties int the templates.
After that, we create an instance of this class and set the needed Properties.
In line 15 we declare the controller which should be used for the IndexRoute.
In Ember a Controller is the component that manages the display logic and the application state.
It is comparable to Knockout's ViewModel object.
As mentioned earlier, each template is backed by a model.
To be more precise, templates retrieve their properties from the controller which decorates the model and provide proxy attributes to avoid for an easier access in the templates. Figure \ref{fig:ember_coupling} visualise how templates, controllers and model are related to each other.
In the IndexRoute object, we finally point to our model instance.

\begin{figure}
	\centering	\includegraphics[width=0.45\textwidth]{./img/tech-ana/ember_controller_coupling.png}
	\caption{Ember model coupling, source \autocite{tech-ana:em-coupling}}
	\label{fig:ember_coupling}
\end{figure} 
 

Per default it is not necessary to define a controller for the IndexRoute, but by convention Ember assumes that the IndexController manages a list of model objects and therefore expects an array for the model property.
Another way to achieve the same thing, is demonstrated in the following listing.
We simply define a special Ember Array and add our created Person instance to it, and finally point the IndexRoute to the array instead.
Since we now have an array as our model, we need to adjust the template, to iterate over all existing array members (see listing \ref{em_array_template}).

\begin{lstlisting}[caption=app.js]
var persons = Ember.A();
persons.pushObject(myself);

App.IndexRoute = Ember.Route.extend({
    model: function() {
        return myself;
    }
});

\end{lstlisting}     

\begin{lstlisting}[label=em_array_template,caption=index.html]
<script type="text/x-handlebars" data-template-name="index">
	{{#each}}
		<div>
			<label>First Name</label> 
			<input {{bindAttr value=firstName}}></input>
		</div>
		<div>
			<label>Last Name</label> 
			<input {{bindAttr value=LastName}}></input>
		</div>
		<p>{{fullName}}</p>
	{{/each}}
</script>
\end{lstlisting}

If we start that application we see that the input fields are properly bound to the model properties.
But if we change the first name or last name in the input field these changes are not reflected to the p-element at the bottom of the page.
This is because Ember does not know how to react on events that are fired on user reaction.
If we want to react on user events and enable a two way data binding,  we need to use an Ember View instead. In Ember, views
\enquote{[...] are responsible for responding to user events, like clicks, drags, and scrolls, as well as updating the contents of the DOM when the data underlying the view changes.} \autocite{tech-ana:em-views}.
Luckily, Ember has a small set of built in views which are Ember.Checkbox, Ember.TextField, Ember.Select and Ember.Textarea.
If necessary, it is also possible to create custom views.
Therefore the only thing we need to change is to replace the input element with an Ember.TextField view as demonstrated in listing \ref{em_view}

\begin{lstlisting}[label=em_view,caption=Ember Views]
	{{view Ember.TextField valueBinding="firstName"}}
\end{lstlisting}

As mentioned earlier, Ember also provides a way to automatically connect Ember Models to a REST API. However, an explanation of this module would be out of the scope of this chapter, hence no furhter basic concepts are involved. Furthermore, using ember data has some crucial disadvantages which are discussed in section \ref{chap:fw-discussion}.

\subsubsection{Angular JS}  

AngularJS \todo{einheitliche Schreibweise, AngularJS überschrift hat es ein Leezeichen. Siehe auch unten, es kommt z.B. auch mal "angular" vor} is developed by Google and has become one of the more popular Frameworks.
AngularJS follows a slightly different approach compared to the other frameworks.
This approach is best described by the Angular developers itself: \enquote{Angular is what HTML would have been had it been designed for applications} \autocite{tech-ana:ng-intro}.
One aspect that shows the difference of Angular are the so called directives, which are a key concept of AngularJS.
Directives allow the user to extend the native HTML with new and application dependent functionality.
This is done by introducing new HTML tags or attributes, as well as some JavaScript code that defines the behaviour and a way how to convert the new introduced tag and the current model state into plain old HTML. Using directives extensively, HTML can be turned into a declarative domain specific language (DSL).

%The second aspect Angular differs from the other frameworks is the templating system.
%(angular docu says that the template system is DOM based ?)   

Furthermore AngularJS also provides two way data binding, filters, routing and dependency injection.
All these components and features make it easy to write Single Page Web Applications without writing any sort of boilerplate code, assuming the architecture and concepts of Angular JS are well known.

%(this is one the disadvantages, takes much time to get familiar with angular)  

%- is the one with the most community activity (github forks, stars, watches) 

Listing \ref{angularBasic} shows the very simple implementation of the same app used before.
The special here is that it is not necessary to write one line of JavaScript code to get things work \todo{working}.
Unfortunately this example isn't very intuitive and it seems that there happens a lot of magic behind the scenes.
In the following sections the details of the example and the important concepts of AngularJS are explained.

\begin{lstlisting}[label=angularBasic,caption=First Angular application]
<html ng-app>
    <head>
        <meta charset="utf-8">
        <title>My AngularJS App</title>
    </head>
    <body ng-init="firstName='Daniel'; lastName='Meiers';">
        <div>
            <label>First Name</label> 
            <input ng-model="firstName"></input>
        </div>
        <div>
            <label>Last Name</label> 
            <input ng-Model="lastName"></input>
        </div>
        <p>Hello {{firstName}} {{lastName}}</p>
        <script src="lib/angular/angular.js"></script>
        <script src="js/app.js"></script>
    </body>
</html>
\end{lstlisting}

The first thing to mention here is the ng-app directive.
This directive defines the application root and Angular automatically looks for this directive during page load.
If found, it starts bootstrapping the angular application which basically means three things:
\begin{itemize}
	\item load the module associated with the directive.
	\item create the application injector
	\item compile the DOM treating the ng-app directive as the root of the compilation.
\end{itemize}

Using this special directive as the root of the Angular related content has the advantage that only a part of the total application can be handled by Angular and the context of the Angular application can be restricted to a part of the DOM.
This makes it much easier to migrate Angular into existing projects or to combine Angular with other frameworks.

In Angular the application consists of one or more so called Modules.
Modules \enquote{[...] declaratively specify how an application should be bootstrapped} \autocite{tech-ana:ng-module}.
This is necessary since Angular applications does not have a main method which can make the instantiation process.
The Module that should be loaded can be referenced in the ng-app directive.\\[5mm]

\begin{lstlisting}[label=ngModule,caption=ng-app declaration with module]
<html ng-app="myApp">
....
<script type="text">
	var myAppModule = angular.module('myApp', []);
</script>
....
\end{lstlisting}

The first parameter of the module creation (line 4 of listing \ref{ngModule}) defines the name of the module that must be used for referencing it in the ng-app directive.
The second parameter can be used to define other modules as dependency for this module.
These dependent modules are created first and then injected into the module per Dependency Injection \todo{Ist der Begriff wirklich ein Name, wegen großschreibung?}.
Among others,the most notably advantage of this approach is, that the separation in modules simplifies unit testing, since not all modules must be loaded for unit testing and additional modules can be loaded, that, can provide some mock up functionality for testing.
This helps writing end-to-end test for the application and allows easily testing the GUI behaviour with unit tests. 

In our simple application (see listing \ref{angularBasic}) we have not defined a module that should be loaded. 
Therefore angular loads a default module.
This default module initializes the application wide Injector which is responsible for the Dependency Injection.
The last step of bootstrapping comprises the re-compilation of the DOM.
Each found Directive or binding expression is evaluated and translated into plain HTML.

The next directive we used in \ref{angularBasic} is the ng-init directive.
this directive allows to do initialisation tasks before the app starts running.
The ng-init directive automatically creates a new property on the corresponding scope object and sets the respective initial values for it, that are declared in the expression. A more detailed explanation of the scope concept is given later on, but in short, the scope in Angular can be compared with the ViewModel object in Knockout.
The ng-model directive binds those properties to the element it is defined in, which automatically enables a two-way data binding between the model properties and the view.
The automatic creation of model properties, is an important difference compared to Knockout and Ember, where this needs to be done manually.  
Before explaining this feature in more detail it is necessary to have a closer look on Angualars scope concept which also requires to understand the meaning of Models, Views and Controllers in the Angular architecture.

The Scope is an object that refers to the application model and provides necessary context for Expressions \todo{klein?} and directives.
Angular expressions are JavaScript-like code snippets that are usually places in data-bindings and can be used for accessing model properties and some basic calculations.
Scopes can watch Angular expressions and propagate events.
These features are used also internally to implement the two way data binding.
Scopes are the "glue between application controller and the view" \autocite{tech-ana:ng-scope} \todo{enquote} and are comparable to Knockouts ViewModel.
An Angular application always consists of exactly one RootScope and multiple ChildScopes.
They are hierarchical nested and resemble the DOM structure as figure \ref{fig:angular_scopes} depicts.
Similar to the model declaration, this can also happen implicit by directives like the ng-repeat directive in figure \ref{fig:angular_scopes}.
ChildScopes prototypically inherit from their parent scope.
Hence, if a binding expression like \{\{firstName\}\} is evaluated, Angular first checks if the Scope object related to that element contains this property.
If not it checks all parent scopes until the property is found or the RootScope is reached.

\begin{figure}
	\centering	
	\includegraphics[width=0.8\textwidth]{./img/tech-ana/angular_scopes.png}
	\caption{Hirarchie of Angular scopes, source \autocite{tech-ana:ng-scope}}
	\label{fig:angular_scopes}
\end{figure} 

In Angular a Model can be any JavaScript object including Arrays \todo{besser klein?} and primitives.
The only condition is that it must be referenced by a Scope \todo{besser klein?} object.
The name of the scope property is also the model identifier that can be used for accessing the value in expressions.
The creation of models can be done explicit or implicit.
In listing \ref{angularBasic} the model is created implicit with the ng-model directive.
To create a model explicit it is necessary to add new properties to the scope in the corresponding Controller.

As already mentioned, Controllers are also a way to create models in Angular and adding them to the Scope.
Controllers are JavaScript functions that are used to augment the Angular scope and are normally used to set up initial state of the Scope or to add application behaviour to it.
Listing \ref{angularController} depicts the needed changes.
\todo{Irgendwie verhindern dass das Listing auf zwei seitein ist und ein Bild dazwischen kommt. Ist bei mir grad der Fall.}
\begin{lstlisting}[label=angularController,caption=Angular Controller]
<html>
....
<body ng-controller="HelloController">
....
<script type="text">
	function HelloController($scope){
	    $scope.lastName = "Meiers";
	    $scope.firstName = "Daniel";
	};
</script>
...
\end{lstlisting}

There are two ways how Controllers are associated with Scope objects.
The first one is to use the ng-controller directive as in the example above.
The second one is to use Angulars Routing mechanism.
Another important thing to mention here \todo{here würde ich einfach streichen} is that the Scope object is provided by Angulars DI (Dependency Injection) system.
  

The last component to cover all elements of the MVC pattern is the View.
In Angular the view consists of the loaded and rendered DOM after Angular has transformed Angular specific directives and expressions into normal HTML.
Figure \ref{fig:angular_views} are giving a good overview over this process.
  
\begin{figure}
	\centering	
	\includegraphics[width=1.0\textwidth]{./img/tech-ana/angular_view.png}
	\caption{Functionality of Angular Views, source \autocite{tech-ana:ng-concepts}}
	\label{fig:angular_views}
\end{figure} 

There are a few more concepts in Angular that helps and simplifies SPA development with Angular.
A really useful feature that makes Angular outstanding are Filters.
Filters are a way to format data that should be displayed to the user.
For example, if we want to print the fullName property of our basic example in uppercase letters we just need to use the Uppercase filter provided by Angular (see listing \ref{angular_filters}).
Angular already provides a basic set of very useful Filters among them Filters for ordering or filtering lists of data but it is also possible to create custom ones.
Filters are used in Angular expressions with a pipe like syntax.

\begin{lstlisting}[label=angular_filters,caption=Angular Filters]
 {{ fullName | uppercase}}
\end{lstlisting}

The last concept are Services.
Services in Angular apps are substitutable objects that are wired together using dependency injection \todo{Falls vorher gross dann auch jetzt.}. Angular Services are implemented as singletons and are used to carry out specific tasks common to web apps and reusable business logic independent of views. Angular provides an impressive set of services such as the \$http service that provides low level access to the browser's XMLHttpRequest object and can be used to make AJAX requests.

The routing mechanism is also implemented as service. This has the advantage that it can be easily injected by dependency injection in every module that wants to use the routing mechanism. The routing service tries to map the path of the current location to an existing route definition. Listing \ref{lst:ng-routes} demonstrates that a route definition basically consists of the template that should be loaded and the controller function the route is related with. The url the route is defined for can also contain dynamic parts, for example the bookId field, that indicates the controller which book object shall be displayed. \todo{m.E. zu langer Satz außerdem beginnt er mit zwei Nomen. Nebensatz mit Komma einleiten}

\begin{lstlisting}[label=lst:ng-routes,caption=Route definition in AngularJS]
var mod = angular.module('ngViewExample', ['ngRoute'])
 
mod.config(function($routeProvider, $locationProvider) {
  $routeProvider.when('/Book/:bookId', {
    templateUrl: 'book.html',
    controller: BookCntl
    }
  });
  $routeProvider.when('/Book/:bookId/ch/:chapterId', {
    templateUrl: 'chapter.html',
    controller: ChapterCntl
  });
  
  function BookCntl($scope, $routeParams) {
	....
  }
});
\end{lstlisting}

\subsection{Discussion} \label{chap:fw-discussion}


After giving a first introduction into the concepts of Knockout.js, Ember.js and AngularJS, \todo{Schreibweise} it is necessary to discuss the pros and cons \todo{Abkürzungen zu informell?} of the different frameworks.

Knockout JS very reduced feature set can be both, a blessing and a curse and it depends on the needs to decide what of these two things prevails.
Knockout JS is a very easy to learn library hence there are only a few concepts to learn.
Another benefit of Knockout JS is, that it has a widely spreaded browser support also for older browser versions.
This benefit is not that much relevant since the applications to develop should be rendered in the JavaFX WebKit component.
Knockout is a very flexible framework hence there are  no regulations for defining the structure and architecture of your \todo{your würde ich hier vermeiden} application.
The missing features of Knockout JS like routing can be compensated with one of the other dozen available third party libraries for this feature.
Unfortunately, without these additional dependencies Knockout JS is rather less suited for building more complex web applications in fact that it forces you to implement too many things manually.
And the additional amount of dependencies could lead to inconsistency problems, increases the error likelihood and the maintenance efforts and make the development even harder (cf \autocite{heise_knockout}).  

Ember however, covers all needed aspects of developing SPA's. \todo{SPAs}
It is possible to write applications with a minimal amount of code, and its concept scale very well, which means it is very easy to write simple applications with Ember that reduces the amount of boilerplate code to a minimum, but the underlying concepts cover also the possibility to create large and complex Single Page Web Applications. With the concept of Ember Components, it is possible to develop reusable components that are furthermore planned to support the upcoming web components standard. 
The only disadvantage is, that Ember is a very strict and opinionated framework.
This is especially true for ember-data which relies heavily on naming conventions to properly work.
Although using approaches like Convention over Configuration is not a principal disadvantage, the question is, if the very stringent corset of Ember will limit further developments or produces a lot of efforts to work around the built in framework concepts.
Ember-data is a good example for that.
Ember-data makes highly use of naming conventions and makes assumptions on the structure of the REST API of the backend. Hence the RESTful API is already defined, it is very likely that the assumptions of ember-data will not correlate with cids REST API, which means that the concepts of the framework need to be work around \todo{nicht sicher ob das so richtig ist. need to be worked around evt?}.
It can not be ruled \todo{out} that similar problems could also arise in other parts of the framework, for example the routing mechanism.

At this point, the concepts of dependency injection, and services, makes Angular much more flexible.
Similar to Ember, Angular is also capable of building all kind of applications, from the simple to the complex one.
From special interest is the very central concept of directives, which can be used to easily develop reusable components. The angular developers claim, that they will support compatibility of directives to the upcoming web component standard.
The large number of concepts makes it a bit harder to get familiar with Angular, but if the concepts are clear, it is very intuitive to build applications with Angular.
Outstanding points of Angular are besides directives, the usage of Dependency Injection, which allows to easily reuse developed Filters, Services and Modules. Additionally the dependency injection increases the testability of the applications and gives the possibility to easily write end to end unit tests. 

Regarding the performance of Angular and Ember there are some important notes to mention. As outlined in chapter \ref{chap:web_dev} Angular and Ember follow two very different approaches for achieving data binding.
In Ember it is necessary to use special objects for the model. Those objects makes it possible to notify about changes.
In Angular it is possible to bind to arbitrary JavaScript objects. It uses dirty-checking to test if any of the bound objects has changed.
The two approaches have some fundamental consequences.
The first and most obvious one is that using dirty checking means that it is possible to bind directly to native JavaScript objects.
This makes the binding expressions slightly more intuitive since no get or set functions are needed to access the properties.
Furthermore it is not necessary to wrap object into  framework depended observable object.
On the other side, the general approach to periodically check the bound JavaScript objects for changes seems to be time consuming and slow.
Although this is conceptually true, this statement needs to be qualified in the practical context.
A good discussion regarding this issue can be found at \autocite{tech-ana:ng-binding}.
The main arguments mentioned there are, that dirty checking is only problematic for a large number of objects and that the human eye can not recognize changes faster than a 50 ms.
There need to be a large amount of objects to check before the 50 ms are exceeded and the user can recognize a sluggish UI.

In addition, also Ember has performance bottlenecks. 
In \autocite{binding_comparison}, Marius Grundersen reveals and discusses the weaknesses of Ember and Angular and demonstrates them with a simple example. 
Figure \ref{fig:demo_app} demonstrates the structure of the demo application. It mainly conists of a input fiels which binds the inserted text to a label above. 
Furthermore, multiple thousands items are added by clicking a button. 

\begin{figure}
	\centering	
	\includegraphics[width=0.8\textwidth]{./img/tech-ana/performance_demo_app.png}
	\caption{Demo Application, source \autocite{binding_comparison}}
	\label{fig:demo_app}
\end{figure} 

Executing this example in Ember reveals that it takes a considerably time until all list items are rendered.
Figure \ref{fig:performance_demo_app} demonstrates the time that is needed for this example. 
Important to note is, that Angular is much faster in that case despite the dirty checking approach. 
The reason for this is, that in Ember it is necessary to set up the notification mechanism for the 5000 objects.
To be fair, the disadvantage of dirty checking first appears when typing in the text field after the list is rendered. 
In this case Angular needs to check all 5000 items although only 1 has changed. 
This makes typing a bit sluggish. 
This example impressively demonstrates that both approaches can have bottleneck effects and it depends on the developer to regard the performance when developing the application. 
A simple solution to avoid the bottleneck affect in Angular is to use multiple scopes.

\begin{figure}
	\centering	
	\includegraphics[width=0.8\textwidth]{./img/tech-ana/performance_comp.png}
	\caption{Performance Comparison of Demo Application, source \autocite{binding_comparison}}
	\label{fig:performance_demo_app}
\end{figure} 

Another important part that is also listed in the requirements in chapter \ref{chap:requirements} is the documentation, the community support and the existence of test tools. 
But also here the difference between Ember and Angular are more from a marginal nature. \todo{Der satz gefällt mir nicht, besonders der Ausdruck: are more from a marginal nature}
The documentations are both very extensive and from a good quality. 
There a many examples that allow to easily use the frameworks. 
Furthermore the documentation of  Ember as well the Angular documentation both provide tutorial and How-To articles. 
However, from a subjective point of view, it appears the the Angular documentation has a slight advantage, which may be due to the fact that Ember is the much younger framework. 
The maturity of Angular can also be the reason for the larger community and popularity of the framework. 
But again, for both frameworks it is easy to find information or get help. 

Regarding the support of test and debugging tools it seems that Angular has a slight advantage hence it has a very good built in testability and also provides a debugging tool Batarang. Batarang helps to examine the scope structure the application and helps monitoring and optimzing the performance. Unfortunately Batarang is only available as Google Chrome extension and cannot be used in the JavaFX WebView.

Summing up, Knockout.js can not be used, hence it has a reduced feature set and does not offer a routing mechanism or reusable components. Ember's very stringent and opinionated approach bears the danger that the framework can limit future developments and complicated and sophisticated workarounds are necessary. Hence AngularJS has none of the above mentioned disadvantages, it is clear that AngularJS is the most promising candidate and shall be used for the further considerations. Table \ref{tbl:js-fw-comp} gives an overview over the different features of each framework and allows a more accessible comparison of them. The disqualifying factors are emphasized with with a coloured background. Up to know the conception of the new required feature is outlined, and the most suiting candidate regarding the browser API as well as a JavaScript MVC Framework is evaluated. Now, that all prerequisites are fulfilled it is possible to start with the implementation. \todo{In der Tabelle bei den Kategorien auf Groß- und Kleinschreibung achten. GitHub z.B.}



\label{tbl:js-fw-comp}

%\todo{ emph disqualifying factors}
	\begin{tabular}{|m{2.8cm}|>{\centering\arraybackslash}m{3.5cm}|>{\centering\arraybackslash}m{3.5cm}|>{\centering\arraybackslash}m{3.5cm}|}
	\hline
	\rowcolor{gray}
		& \head{Angular JS	} & \head{Ember JS} & \head{Knockout JS}	  \tabularnewline
		\hline
	2-way data binding	 		
		& \checkmark
		& \checkmark
		& \checkmark  \\ \hline
	normal and computed properties	
		& \checkmark / change detection
		& \checkmark / \checkmark
		& \checkmark / \checkmark\\ \hline
	Testing				
		& end-to-end tests, build in tools for unit tests, debugging extension for chrome 
		& extra packages ember-testing, qUnit 
		&  unit tests \\ \hline
	Routing				
		& \checkmark 
		& \checkmark	
		& \cellcolor{LightRed} -		\\ \hline
	Multiple and Composite Views	
		& \checkmark / \checkmark
		& \checkmark / \checkmark 
		& \cellcolor{LightRed} -		\\ \hline
	Backend connection		
		& optional (ajax wrapper)	
		& optional (ember-data)	
		& -	\\ \hline
	%Dependency injection
	%	& \checkmark
	%	& -
	%	& -\\ \hline
	%Templating system
	%	& Dom based 
	%	& String based (handlebars)
	%	& Dom based\\ \hline
	Documentation
		& very good
		& good
		& good\\ \hline
	API stability / maturity
		& since 2009, stable
		& since 2011, 
		&since 2010, very stable\\ \hline
	community (github watch/star/fork
		& 1450/13061/3147
		& 672/7752/1568
		& 349/4074/657\\ \hline
	Perfomance
		& slowest when the model is complex
		& slowest when rendering large lists
		& slowest when pushing many items\\ \hline
	Flexibile vs opinonated
		& flexible
		& \cellcolor{LightRed} stringent \& opinionated
		& flexible\\ \hline
	%size
	%	& 80kb
	%	& 56kb + JQuery + Handlebars
	%	& 15kb\\ \hline	
	%FrameWork / Library
	%	& Something in between
	%	& Framework
	%	& Library\\ \hline
	Integration of UI Elements
		& \checkmark use directives
		& \checkmark use Ember views
		& \checkmark use custom bindings \\ \hline
	License
		& MIT License
		& MIT License
		& MIT License\\ \hline
	\end{tabular}



